\documentclass[14pt,a4paper,parskip]{scrbook}
\usepackage[utf8]{inputenc}
\usepackage{amsmath}
\usepackage{amsfonts}
\usepackage{amssymb}
\usepackage{url}
\usepackage{eurosym}
\usepackage[margin=3cm,noheadfoot,top=0.5cm]{geometry}

\usepackage{setspace}
\usepackage{graphicx}


\begin{document}
\chapter*{\centering \textbf{VideLibri}}
\section*{\centering das Ausleihverwaltungsprogramm}

	\thispagestyle{empty}
Bis du es auch leid, ständig Säumnisentgelte zahlen zu müssen?\\
Mit VideLibri ist die Zeit der verpassten Abgabefristen vorbei!

VideLibri kann auf deinem Computer deine Ausleihen aller Zweigstellen der Stadtbüchereien, der Universitätsbibliothek und der Fachhochschulbibliothek von Düsseldorf verwalten.

Damit kannst du Ausleihen von allen registrierten Konten und Bibliotheken gleichzeitig sehen und verlängern, so dass du nicht mehr wie bisher andauernd mehrere Internetseiten aufrufen musst.

Zudem werden die Medien automatisch verlängert, wodurch du dich nie wieder selbst um das Verlängern kümmern musst.\\
 Vor dem endgültigen Ablaufen der Leihfrist wirst du schließlich vom Programm gewarnt.

Desweiteren werden alle jemals ausgeliehenen Medien gespeichert, so dass du immer nachsehen kannst, wann du was ausgeliehen hattest. Mit Hilfe des integrierten BibTeX-Export kann automatisch ein Literaturverzeichnis dieser Medien generiert werden, was für eine Abschlussarbeit sehr nützlich ist.

Für jeden der über fünfzig Bücher ausgeliehen hat, ist dieses Programm ein absolutes Muss. ($\Leftrightarrow$)

VideLibri ist unter \url{http://www.benibela.de/tools_de.html} erhältlich.

\begin{small}
(benötigt eigentlich: Windows, Internet und hundert ausgeliehene Bücher pro Jahr)

(Es ist Shareware und die kostenlose Version hat alle oben genannten Funktionen. Die Vollversion kostet 10 \euro (Selbstkostenpreis), aber diese Ausgabe amortisiert sich nach ein paar Wochen über die entfallenen  Säumnisentgelte. ) 
\end{small}
%
%\begin{minipage}}
%
%\end{minipage}\rotatebox{-45}{\url{www.benibela.de/tools_de.html}}}
%{\rotatebox{-45}{\url{www.benibela.de/tools_de.html}}}
%\rotatebox{-45}{\url{www.benibela.de/tools_de.html}}
%\rotatebox{-45}{\url{www.benibela.de/tools_de.html}}
\end{document}
